%%% Local Variables:
%%% coding: utf-8
%%% mode: latex
%%% TeX-engine: xetex
%%% TeX-command-extra-options: "-shell-escape"
%%% End:

\documentclass[14pt, a4paper]{extarticle}

\usepackage{graphicx}
\DeclareGraphicsExtensions{.jpg,.png}

\usepackage{hyperref}

\usepackage{amsfonts}
\usepackage{amsmath}

\usepackage[english,russian]{babel}

\usepackage{fontspec} 
\defaultfontfeatures{Ligatures={TeX},Renderer=Basic}
\setmainfont[Ligatures={TeX,Historic}]{Times New Roman}
\setmonofont{Courier New}
\newfontfamily\cyrillicfonttt[Script=Cyrillic]{Courier New}
\urlstyle{same}

\usepackage{xcolor}

\usepackage{tabularx}

\usepackage{indentfirst} %отступ первой строки первого абзаца
\linespread{1.25}

\usepackage{geometry}
\geometry{left=3cm}
\geometry{right=1cm}
\geometry{top=2cm}
\geometry{bottom=2cm}

\hypersetup{
    colorlinks,
    citecolor=black,
    filecolor=black,
    linkcolor=black,
    urlcolor=black
}

\usepackage[final]{pdfpages}

\usepackage{titlesec} % оформление заголовков

\titleformat{\section}[block]
	{\newpage\hspace{\parindent}\bfseries\fontsize{18pt}{21.6pt}\selectfont}
        {\thesection}
        {1em}{\MakeUppercase}
\titleformat{name=\section,numberless}[block]
	{\newpage\centering\bfseries\fontsize{18pt}{21.6pt}\selectfont}
        {}
        {1em}{}
\titleformat{\subsection}[block]
	{\bfseries\hspace{\parindent}\fontsize{16pt}{19.2pt}\selectfont}
        {\thesubsection}
        {1em}{}
        
\usepackage{float}
\usepackage{caption}

\usepackage[newfloat]{minted}
\usepackage{fancyvrb}
\newenvironment{code}{\captionsetup{type=listing}}{}


\DeclareCaptionLabelSeparator{emdash}{\;\textemdash\;}
\captionsetup[figure]{name={Рисунок}, labelsep=emdash, justification=centering, singlelinecheck=off, font={small, bf}, labelfont=bf}
\captionsetup[table]{name={Таблица}, labelsep=emdash, justification=raggedright, singlelinecheck=off, font={small, it}, labelfont=it}
\captionsetup[listing]{name={Листинг}, labelsep=emdash, justification=raggedright, singlelinecheck=off, font={small, it}, labelfont=it}

\usepackage{array}
\newcommand\ChangeRT[1]{\noalign{\hrule height #1}}

\usepackage{ragged2e}
\usepackage{microtype}

\justifying
\sloppy
\tolerance=500
\hyphenpenalty=10000
\emergencystretch=3em

\usepackage{setspace}

\begin{document}

\makeatletter
\renewcommand{\l@section}{\@dottedtocline{1}{0em}{1.25em}}
\renewcommand{\l@subsection}{\@dottedtocline{2}{0em}{1.75em}}
\renewcommand{\l@subsubsection}{\@dottedtocline{3}{0em}{2.6em}}
\renewcommand{\@dotsep}{1.25}
\makeatother

\def\contentsname{СОДЕРЖАНИЕ}

%\pagenumbering{gobble}
\begin{titlepage}
\includepdf{title}
\end{titlepage}
%\tableofcontents

\section*{Теоритическое введение}
Язык Java - это объектно-ориентированный язык
программирования. Программы написанные на Java могут выполняться
на различных операционных системах при наличии необходимого ПО -
Java Runtime Environment.

Для того чтобы создать программу на языке Java необходимо
следующее ПО:
\begin{itemize}
\item Java Development Kit (JDK);
\item Java Runtime Environment (JRE);
\item Среда разработки. Например NetBeans или IDE IntelliJ IDEA.
\end{itemize}
Чтобы начать написание программы необходимо запустить среду
разработки. При первом запуске среды обычно нужно указать путь к
JDK, чтобы можно было компилировать код и запускать программу. В
среде разработки необходимо создать Java проект, после чего
необходимо создать пакет и в нем создать какой-либо класс. Также в
свойствах проекта нужно указать класс, с которого будет начинаться
запуск программы.

В классе, с которого будет начинаться запуск программы
обязательно должен быть статический метод main(String[]), который
принимает в качестве аргументов массив строк и не возвращает
никакого значения. 

Чтобы объявить переменную, необходимо указать тип переменной
и ее имя. Типы переменной могут быть разные: целочисленный(long, int,
short, byte), число с плавающей запятой(double, float),
логический(boolean), перечисление, объектный(Object).
Переменным можно присваивать различные значения с помощью
оператора присваивания <<=>>.

Целочисленным переменным можно присваивать только целые
числа, а числам с плавающей запятой - дробные. Целые числа
обозначаются цифрами от 0 до 9, а дробные можно записывать отделяю
целую часть от дробной с помощью точки. Переменным типа float
необходимо приписывать справа букву "f", обозначающую, что данное
число типа float. Без этой буквы число будет иметь тип double.
Класс String - особый класс в Java, так как ему можно присваивать
значение, не создавая экземпляра класса(Java это сделает
автоматически). Этот класс предназначен для представления строк.
Строковое значение записывается буквами внутри двойных кавычек.

С целочисленными переменными можно совершать различные
операции: сложение, вычитание, умножение, целое от деления, остаток
от деления. Эти операции обозначаются соответственно <<+>>, <<->>, <<*>>, <</>>,
<<\%>>. Для чисел с плавающей запятой применимы операции сложения,
вычитания, умножения, деления. Для строк применима операция <<+>>,
обозначающая конкатенацию, слияние строк.

Массив — это конечная последовательность упорядоченных
элементов одного типа, доступ к каждому элементу в которой
осуществляется по его индексу.

Для того чтобы создать массив переменных, необходимо указать
квадратные скобки при объявлении переменной массива. После чего
необходимо создать массив с помощью оператора new. Необходимо
указать в квадратных скобках справа размер массива.

Условие - это конструкция, позволяющая выполнять то или другое
действие, в зависимости от логического значения, указанного в условии. 

Цикл - это конструкция, позволяющая выполнять определенную
часть кода несколько раз. В Java есть три типа циклов for, while, do while.
Цикл for - это цикл со счетчиком, обычно используется, когда
известно, сколько раз должна выполниться определенная часть кода. 

Для ввода данных используется класс Scanner из библиотеки пакетов
Этот класс надо импортировать в той программе, где он будет
использоваться. Это делается до начала открытого класса в
коде программы.

В классе есть методы для чтения очередного символа заданного
типа со стандартного потока ввода, а также для проверки
существования такого символа.

Для работы с потоком ввода необходимо создать объект
класса Scanner, при создании указав, с каким потоком ввода он
будет связан. Стандартный поток ввода (клавиатура) в Java
представлен объектом — System.in. А стандартный поток вывода
(дисплей) — уже знакомым вам объектом System.out. Есть ещё
стандартный поток для вывода ошибок — System.err

Методы позволяют выполнять блок кода, из любого другого места,
где это доступно. Методы определяются внутри классов. Методы могут 
быть статическими(можно выполнять без создания экземпляра класса),
не статическими (не могут выполняться без создания экземпляра
класса). Методы могут быть открытыми(public), закрытыми(private).
Закрытые методы могут вызываться только внутри того класса, в
котором они определены. Открытые методы можно вызывать для
объекта внутри других классов.

При определении метода можно указать модификатор
доступа(public, private, protected), а также указать статический ли метод
ключевым словом static. Нужно обязательно указать тип возвращаемого
значения и имя метода. В скобках можно указать аргументы, которые
необходимо передать методу для его вызова. В методе с непустым
типом возвращаемого значения нужно обязательно указать оператор
return и значение, которое он возвращает. Если метод не возвращает
никакого значения, то указывается тип void.
\section*{Постановка задачи}
\begin{enumerate}
\item Создать проект в IntelliJ IDEA
\item Создать свой собственный Git репозитарий
\item Написать программу, в результате которой массив чисел
создается с помощью инициализации (как в Си) вводится и считается в
цикле сумма элементов целочисленного массива, а также среднее
арифметическое его элементов результат выводится на экран. Использовать
цикл for.
\item Написать программу, в результате которой массив чисел вводится
пользователем с клавиатуры считается сумма элементов целочисленного
массива с помощью циклов do while, while, также необходимо найти
максимальный и минимальный элемент в массиве, результат выводится на
экран.
\item Написать программу, в результате которой выводятся на экран
аргументы командной строки в цикле for.
\item Написать программу, в результате работы которой выводятся на
экран первые 10 чисел гармонического ряда (форматировать вывод).
\item Написать программу, которая с помощью метода класса,
вычисляет факториал числа (использовать управляющую конструкцию
цикла), проверить работу метода.
\item Результаты выполнения практической работы залить через IDE в
свой репозитарий и продемонстрировать преподавателю.
\end{enumerate}
\section*{Программный код}
\begin{code}
\captionof{listing}{Код для задания 1}
\begin{Verbatim}[frame=single, fontsize=\footnotesize]
import java.util.Scanner;

public class Main {
    public static void main(String[] args) {

        Scanner in = new Scanner(System.in);
        System.out.print("input amount of numbers in array: ");
        int n = in.nextInt();

        int[] array = new int[n];

        System.out.print("input numbers in array: ");
        for (int i = 0; i < n; i++) {
            array[i] = in.nextInt();
        }

        int sum1 = 0;
        int sum2 = 0;
        int sum3 = 0;

        for (int i = 0; i < n; i++) {
            sum1 += array[i];
        }

        int j = n;
        do {
            j--;
            sum2 += array[j];
        }
        while (j > 0);

        int k = 0;
        while (k < n) {
            sum3 += array[k];
            k++;
        }

        System.out.println(sum1);
        System.out.println(sum2);
        System.out.println(sum3);
        in.close();
    }
}
\end{Verbatim}
\end{code}
\begin{code}
\captionof{listing}{Код для задания 2}
\begin{Verbatim}[frame=single, fontsize=\footnotesize]
public class Main {
    public static void main(String[] args) {
        double[] harmonic = new double[10];
        for(int i = 0; i < 10; i++) {
            harmonic[i] = 1. / (i + 1);
        }

        double sum = 0;

        System.out.print("first 10 numbers in a row: ");
        for(int i = 0; i < 10; i++) {
            String result = String.format("%.3f", harmonic[i]);
            sum += harmonic[i];
            System.out.println(result);
        }

        String result = String.format("%.3f", sum);
        System.out.print("sum: ");
        System.out.println(result);

    }
}
\end{Verbatim}
\end{code}
\begin{code}
\captionof{listing}{Код для задания 3}
\begin{Verbatim}[frame=single, fontsize=\footnotesize]
public class Main {
    public static void main(String[] args) {

        Scanner in = new Scanner(System.in);
        System.out.print("input amount of numbers in array: ");
        int n = in.nextInt();

        int[] array = new int[n];
        int[] array1 = new int[n];
        Random random = new Random();
        for (int i = 0; i < n; i++) {
            array[i] = random.nextInt();
        }

        for (int i = 0; i < n; i++) {
            array1[i] = (int) (Math.random() * 10);
        }

        Arrays.sort(array1);
        Arrays.sort(array);
\end{Verbatim}
\end{code}

\begin{code}
\captionof*{listing}{Продолжение листинга 3}
\begin{Verbatim}[frame=single, fontsize=\footnotesize]
        System.out.println("1st array: ");
        System.out.println(Arrays.toString(array));
        System.out.println("2nd array: ");
        System.out.println(Arrays.toString(array1));

        in.close();
    }
}
\end{Verbatim}
\end{code}
\section*{Вывод программы}
Ввод программы:
\begin{code}
\captionof{listing}{Ввод}
\begin{Verbatim}[frame=single, fontsize=\footnotesize]
input amount of numbers in array: 4
input numbers in array: 4 3 2 7
16
16
16
\end{Verbatim}
\end{code}
Вывод программы:
\begin{code}
\captionof{listing}{Вывод задания 1}
\begin{Verbatim}[frame=single, fontsize=\footnotesize]
16
16
16
\end{Verbatim}
\end{code}
\begin{code}
\captionof{listing}{Вывод задания 2}
\begin{Verbatim}[frame=single, fontsize=\footnotesize]
first 10 numbers in a row: 1,000
0,500
0,333
0,250
0,200
0,167
0,143
0,125
0,111
0,100
sum: 2,929
\end{Verbatim}
\end{code}
\begin{code}
\captionof{listing}{Вывод задания 3}
\begin{Verbatim}[frame=single, fontsize=\footnotesize]
input amount of numbers in array: 5
1st array: 
[108930068, 160477114, 264900488, 1765772861, 2140865884]
2nd array: 
[0, 2, 2, 6, 9]
\end{Verbatim}
\end{code}
\section*{Вывод}
В ходе данной лабораторной работы были получены практические навыки разработки программ, изучен синтаксис языка Java, были освоены основные конструкции языка Java и навыки осуществления стандартного потока ввода/вывода данных.
\end{document}